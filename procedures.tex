\chapter{Procedures} 
\label{procedures}  
\section{Evacuating Transfer lines}
Helium transfer lines (TLs) are vacuum insulated tubes.  Typically, one end is plunged directly into a helium dewar while the other plugs into the vessel that receives liquid helium, with a pressure differential driving the helium through.  If the vacuum jacket is compromised, the helium will cool the outer shell of the TL and frost up (helium will of course stop transferring).  Once this happens, the only solution is to remove and warm up the transfer line and evacuate the vacuum space.


TLs are evacuated with a leak checker until the leak rate plateaus.  Generally, this takes three hours (depending on the leak checker and TL), but if time permits there is no harm in letting it pump overnight.  In theory, an evacuated TL will be fine for months, but in practice it is safer to pump down each TL at the beginning of a cooldown in case the vacuum space was somehow compromised in storage/transport.


\subsection{Cryofab TLs}

\subsection{LTL}

\section{Indium Extrusion}
Hifrost requires indium wire precisely 1 mm in diameter for making MC and IVC seals.  Indium seals must be scraped away after use, and the metal is always recovered (due to its scarcity).  The Indium Association of America charges roughly \$1000 for 13 feet, or about \$20-30 per attempted seal, so it is worth creating our own wire from the so-called ``scraps''.

Indium extrusion is the name of the process that transforms bulk indium into nicely shaped wire.

\subsection{Background} 

Indium scraps or ingots are put into a hollow cylinder with a small hole in the bottom, and the indium is softened with heat and pushed through to make a 1 mm wire.  The melting point of indium is about $150 ^\circ$ C, but approaching this temperature leaves the indium ``runny'' and impossible to shape.  We have found heating the cylinder to $100 ^\circ$ C with a heating tape yields the ideal consistency.

Our first indium die, the plate at the bottom of the hollow cylinder that has a small hole for shaping indium, had a 1 mm hole.  For whatever reason, this produced indium 0.75 mm in diameter (too small for the HiFrost flanges).  We increased the die to 41 mils (about 5\% larger), and the indium wire came out at 1 mm.  

When extruding the indium, a varying force on the hydraulic pump sometimes leads to uneven wire (something like sausage links).  To prevent this, a continuous, non-varying force should be applied for an entire press.

\subsection{Materials}
The \textbf{extrusion cylinder} is an aluminum cylinder with six 1/4-20 bolt holes tapped in the bottom.  The inside of the aluminum is lined with another hollow cylinder made of steel, which precisely matches a \textbf{steel rod} that slides in and out.  If the rod and inner cylinder do not precisely fit, either the rod will not fit in the cylinder or the indium will leak out during the press.  There is a small hole drilled in the side of the aluminum for a thermocouple.

Screwed into the bottom of the extrusion cylinder is the \textbf{indium die}, a plate with a hole through the middle to push indium through.  The entrance hole in top of the die matches in the extrusion cylinder, and the exit hole in the bottom of the die shapes the indium wire.  The die is a separate component from the extrusion cylinder for cleaning and modularity reasons.

\textbf{Indium} is generally recovered from scraps of old seals or ingots are bought from the Indium Association of America (currently a pound costs \$300).  Ingots should be cut with a knife into smaller blocks to fit inside the extrusion cylinder.

The \textbf{extrusion stand} is a mounted, horizontal bar for the hydraulic press to push against so pressure is applied to the extrusion cylinder.  There must also be a hole in the base of the stand for the pressed indium wire to be collected.

A \textbf{thermocouple} sensor is used with an accompanying \textbf{thermocouple readout} to measure the temperature of the extrusion cylinder.

\textbf{Heat tapes} warm the indium so it may be pressed.  The tapes should be long enough to wrap around the cylinder to warm it evenly.

A layer of \textbf{aluminum foil} is wrapped around the heating tapes to keep the heat from dissipating away.

\textbf{Latex and clothe gloves} are for personnel safety and keeping the indium sanitary.

A \textbf{variable alternating current} unit adjusts the temperature of the heating tape.

The indium is extuded onto sterile \textbf{\com{Absorbond Wiper Sheets, TX}} 9\inches x 9\inches, 409, and kept for storage.

An \com{Enerpc} P39 Porta Power \textbf{hydraulic hand pump} provides the force for extrusion.

\subsection{Extrusion process}

\begin{figure}
 \centering
 \includegraphics[scale=0.5]{img/extruder-setup.png}
 % extruder-setup.png: 359x275 pixel, 72dpi, 12.66x9.70 cm, bb=0 0 359 275
 \caption{Indium extruder setup: A) extruder stand, B) hydraulic pump, C) steel rod, D) extrusion cylinder, E) indium scrap/ingot fragments, F) finished wire exiting die, G) sterile wipe sheet, H) hand jack, I) hydraulic line, J) thermocouple gauge}
 \label{fig:extruder}
\end{figure}

\begin{figure}
 \centering

\begin{minipage}[b]{0.45\linewidth}
 \includegraphics[width=\linewidth]{img/extruder-setup-photo.png}
 \caption{A photo of the setup.}
 \label{fig:extruder-setup-photo}
\end{minipage}
\quad
\begin{minipage}[b]{0.45\linewidth}
 \includegraphics[width=\linewidth]{img/extruder-cylinder.png}
 % extruder-setup.png: 359x275 pixel, 72dpi, 12.66x9.70 cm, bb=0 0 359 275
 \caption{The extruder cylinder without the die.  The steel rod is plunged all the way down and leftover indium can be seen on the surface.}
 \label{fig:extruder-cylinder}
\end{minipage}
\end{figure}


\begin{enumerate}
\item Prepare electrical connections for heating tape and thermocouple gauge and readout.
\item Remove steel rod from extrusion cylinder and insert blocks of indium.  Place the steel rod back in the cylinder.  \textit{NB: after losses, 100 grams of indium provides approximately 10 feet of wire instead of the theoretical 50 feet.}
\item Wrap the cylinder with heating tape and aluminum foil.
\item Place the cylinder/rod under the extrusion stand.
\item Set the variable current source so the heating tape reaches about $70^\circ$ C, which takes around 20 minutes.  Increase the temperature in steps, allowing the thermocouple to equilibrate between steps, to $100 ^\circ$ C.  \textit{NB: on our VariAC unit the 50 mark corresponds to $75 ^\circ$ C and the 60 mark corresponds to $100^\circ$ C}
\item Set the hydraulic jack on the extrusion cylinder and under the stand's horizontal bar.
\item Use the hand jack to slowly push the indium through the extrusion cylinder and out the die.  The press takes about 5 minutes.  A second person slowly winds the indium wire on the wiper sheets. 
\end{enumerate}


\begin{figure}
 \centering
 \includegraphics[scale=0.75]{img/extruder-demo.png}
 % extruder-setup.png: 359x275 pixel, 72dpi, 12.66x9.70 cm, bb=0 0 359 275
 \caption{The hydraulic press sitting on top of the extrusion cylinder.}
 \label{fig:extruder-demo}
\end{figure}


\section{Making Indium Seals}

\section{Breaking Indium Seals}
\begin{enumerate}
 \item Remove screws with 1.5 mm allen key and the modified MC wrench.  Visually inspect each screw for indium or signs of stripping, and turn each nut back on the screw it came off of.  If there is any torque resistance or sign of stripping, the screw was likely damaged when the indium seal was tightened and both the screw and nut should be discarded.  If there is no damage, return the nuts/screws to the bag labeled ``MC Screws and Nuts''.
\item Use two MC screws and the 1.5 mm allen key as jacking screws the break the indium seal.  Defining the top of the fridge (in the horizontal orientation) as 12:00, the jacking screw holes are at 1:00 and 5:00 look upstream.  The jacking screw holes can be identified by threaded holes in the MC bell housing that have no matching through holes on the fridge MC flange.  Alternate between screws making 1/4 turns until the MC is free.
\item Remove the MC from the fridge, careful not to damage the sensors on the copper dam.
\item Remove the jacking screws from the MC and place in the bag labeled ``MC Screws and Nuts''.
\item Scrape indium, if any, off the MC and place the MC in the safety zone.
\item Scrape indium off the fridge flange, recovering as much as possible in the indium scrap container; be very careful not to touch the stainless steel beam window.
\end{enumerate}


\section{Magnet Lead Installation}

\section{Mounting Fridge}

\section{Dismounting Fridge}

\section{Cleaning Viton/Buna O-rings and Grooves}

\section{Changing Kenol Connectors}

\section{Installing Heating Tapes}

\section{Filling \lnn{} Trap}

\section{Purging Helium Lines}

\section{Vacuum Rise Testing}

\section{Stripping Fridge}

\section{Assembling Fridge}

When putting the fridge together, there are two primary concerns: forgetting to install something (e.g., superinsulation, microwave guide support) and leaky seals.  This section does not include a cold target load and assumes the fridge is hanging vertically on the fridge stand in the Vault.

\subsection{MC}
\paragraph{Tools}
\begin{itemize}
 \item qtips/wire cutters/isopropyl alcohol
\item indium scrap box
\item 1.5 mm allen key
\item MC open faced allen wrench (4.0)
\item MC
\item 1 mm indium wire
\item chemistry clamp
\item TODO sizes for MC nuts and bolts
\end{itemize}

Read the section on making indium seals.

Hold the MC in place with the chemistry clamp so the bell housing is about 1 cm from touching the fridge flange.  Place the bolts through the MC holes and gently tighten the nuts one or two turns so the bolts don't fall.  Inspect the indium wire to make sure it is still in place, then tighten the nuts in a ``star'' pattern (never tightening adjacent screws consecutively).  Hold the bolts with an allen key while turning the wrench so the heads do not strip.  Use only the special MC wrench so the fridge does not scratch.

\begin{figure}
 \centering
 \includegraphics[scale=.75]{img/mc-stand.jpg}
 % mc-stand.jpg: 560x420 pixel, 300dpi, 4.74x3.56 cm, bb=0 0 134 101
 \caption{The MC with a prepared indium seal behind held by a chemistry clamp.}
 \label{fig:mc-stand}
\end{figure}


If loading conditions permit, leak check the MC before moving on.

  \subsection{Microwave Guide Support}
\paragraph{Tools}
\begin{itemize}
 \item small flat head screwdriver
\item squeeze-to-open tweezers
\end{itemize}

Three brass screws are usually stored in the threaded struts on the fridge where they fasten the waveguide support in place.

After unscrewing them, raise the microwave guide support over the MC making sure the waveguide is lined up with both the slot in the MC bellhousing and the circular waveguide connector on the fridge.  One person holds the support while another tightens the screws.  Hold the screws with tweezers to prevent dropping them down towards the MC (if this happens, you must start over, careful not to damage the fridge with the loose screw while taking off the support).  Warning: the microwave guide support can still fall off when it is being held by one screw.

Make sure the waveguide (or the support) is not touching the MC.  Normally this will not happen unless the support has been deformed.

  \subsection{IVC}

\paragraph{Tools}
\begin{itemize}
 \item solder station
\item 2.5 mm allen key
\item indium scrap box
\item qtips/wire cutters/isopropyl alcohol
\item TODO find out IVC screw sizes
\end{itemize}

Read the section on making indium seals.

Similar to installing the MC, first prepare the IVC indium seal and place it in the groove on the IVC flange.  One person holds the IVC within about 1 cm from the fridge while another puts in the screws.  Initially, only turn in 2 diametrically opposite screws 1 or 2 turns each so they support the IVC and it no longer needs to be held by another person.  Turn in the other 4 screws 1 or 2 turns each.  Inspect the indium seal to make sure it is still in place (it should not be touching the fridge flange yet) before turning in the screws in a ``star pattern'' similar to the MC.  Be careful not to strip the screw heads.


Leak check if possible.


  \subsection{OVC}

\paragraph{Tools}
\begin{itemize}
 \item pliers for pilot pins
\item 5 mm allen key
\item three OVC cans (inner, outer, nose cone)
\item triple flange orings
\item pilot pins
\item 2x nut plates
\item triple flange screws TODO find size
\end{itemize}

Make sure the orings and grooves are clean (see cleaning orings section).  Place the outer OVC can on the blue fridge lift in the Vault, and lower the inner OVC can inside of it, making sure the black lines on the flanges line up.  Place the pilot pins up through the pilot holes and secure the cans in place with nuts and threaded rod that fit through the fridge lift platform.  Hook up the OVC manifold to the pumpout (making sure the hose is coming off at the appropriate angle for Blowfish clearance) and warm up the OVC turbo pump.  Raise the nose cone in place, being sure to line up the large dents, and begin pumping.  Leak check the OVC.


With the OVC still being pumped, remove it from the fridge lift and raise it over the fridge.  Line up the black lines on the triple flange and push the pilot pins in by hand, using pliers if they need to be rotated or pulled out slightly.


Loosely turn in all the triple flange screws, careful not to strip the aluminum taps.  Tighten in a star formation so the oring does not become unevenly compressed.

\section{Removing/Replacing \het{} Baffles}